% !TeX root = ./tesi.tex
% !TeX encoding = UTF-8 Unicode
% !TeX spellcheck = it_IT
% !TeX program = arara
% !TeX options = --log --verbose --language=it "%DOC%"

% arara: lualatex:      { interaction: batchmode }
% arara: frontespizio:  { interaction: batchmode, engine: lualatex }
% arara: biber
% arara: lualatex:      { interaction: batchmode }
% arara: lualatex:      { interaction: nonstopmode, synctex: yes }

\documentclass[%
  a4paper,                % formato di pagina A4
  12pt,                   % corpo del testo a 12pt
  % la dimensione 12pt automaticamente imposta \footnotesize a 10pt
  twoside,                % (oneside|twoside) documento a singola o doppia facciata,
  openright,              % (openany|openright) fa cominciare un capitolo nella successiva pagina a disposizione o sempre in una pagina destra
  % twocolumn,            % dà a LaTeX le istruzioni per comporre l'intero documento su due colonne
  titlepage,              % (titlepage|notitlepage) se dopo il titolo del documento debbaavere  inizio  una  nuova  pagina
  % fleqn,                % allinea le formule a sinistra rispetto a un margine rientrato
  % leqno,                % mette la numerazione delle formule a sinistra anziché a destra
  final                   % (draft|final) scelta tra bozza o finale, influenza il comportamento degli altri pacchetti
]{scrbook}

\usepackage{fancyvrb}       % fornisce l'ambiente VerbatimOut e modifica listati di codice

\begin{VerbatimOut}{\jobname.xmpdata}
\Title{Titolo}
\Subject{Oggetto}
\Author{Niccolò Maltoni}
\Copyright{Questo documento è fornito sotto licenza Apache License, Version 2.0}
\CopyrightURL{https://opensource.org/licenses/Apache-2.0}
\end{VerbatimOut}

\usepackage{unibotesi}

\begin{document}

  \frontmatter{}
  \pagenumbering{Roman}
  \pagestyle{empty}
  % !TeX root = ../../tesi.tex
% !TeX encoding = UTF-8 Unicode
% !TeX spellcheck = it_IT

\begin{Preambolo*}
  \usepackage{fontspec}
  \setmainfont[Ligatures=TeX]{Latin Modern Roman}
\end{Preambolo*}
\begin{frontespizio}
  \Universita{Bologna}        % aggiunge da sé “Università degli Studi di”.
  \Istituzione{%
    Alma Mater Studiorum --- Università di Bologna \\%
    Campus di Cesena%
  }
  \Divisione{Dipartimento di Informatica --- Scienza e Ingegneria}
  \Corso[Laurea magistrale]{Ingegneria e Scienze Informatiche}
  \Annoaccademico{2019--2020}
  \Titolo{Titolo}
  \Sottotitolo{Tesi in \ldots}
  % \Preambolo{\renewcommand{\frontsmallfont}[1]{\small}}       % non viene stampata la matricola
  % \Preambolo{\renewcommand{\frontsmallfont}[1]{\small Matr.}} % abbrevia la matricola
  \Candidato[MATRICOLA123456]{Niccolò~Maltoni}
  \NCandidato{Presentata da}  % sostituisce la parola “Candidato”
  \Relatore{Prof.~Tizio~Relatore}
  \Correlatore{Prof.~Tizio~Correlatore}
  \Piede{%                    % sostituisce la scritta “Anno Accademico” nel piede
    III sessione di laurea \\%
    Anno Accademico 2019--2020%
  }
\end{frontespizio}

% Necessario per Overleaf: compila il TeX del frontespizio subito dopo averlo generato
\IfFileExists{\jobname-frn.pdf}{}{%
\immediate\write18{lualatex \jobname-frn}}

  % !TeX root = ../../tesi.tex
% !TeX encoding = UTF-8 Unicode
% !TeX spellcheck = it_IT

\clearemptydoublepage{}
\thispagestyle{empty}
\vspace*{20ex}
\begin{flushright}
    \begin{LARGE}
        \textbf{Parole chiave}\\
        \vspace{5ex}
    \end{LARGE}
    \begin{normalsize}
        \textbf{%
            Parola chiave 1\\%
            \medskip
            Parola chiave 2%
        }
    \end{normalsize}
\end{flushright}
\vfill

  % !TeX root = ../../tesi.tex
% !TeX encoding = UTF-8 Unicode
% !TeX spellcheck = it_IT

\clearemptydoublepage{}
\null{}\vspace{\stretch{1}}
\begin{flushright}
    \textit{Dedica}
\end{flushright}
\vspace{\stretch{2}}\null{}

  % !TeX root = ../../tesi.tex
% !TeX encoding = UTF-8 Unicode
% !TeX spellcheck = it_IT

\begin{abstract}
  Abstract
\end{abstract}

  \tableofcontents

  \mainmatter{}
  \pagenumbering{arabic}
  \pagestyle{headings}
  Minimo documento
  \appendix
  \begin{appendices}
  Appendice
\end{appendices}


  \backmatter{}
  \nocite{*}            % aggiunge tutti i riferimenti nel .bib (anche non citati)
\printbibliography[%  % produce la bibliografia
  heading=bibintoc    % inserisce il titolo nell'indice generale
]

  \input{src/3-backmatter/ringraziamenti.tex}

\end{document}
